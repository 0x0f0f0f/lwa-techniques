\section{Introduction}
\label{sec:intro}

Weighted automata (WAs) are a generalization of non-deterministic automata.
When reading a symbol, a non-deterministic automaton 
can transition in different states simultaneously. Weighted automata
introduce \textit{weights} over transitions, which can for example, 
represent the cost of a transition, 
or in probabilistic systems, the chance of such transition to happen.
WAs can be represented with a set of states, an output function and
a set of transition matrices, indexed over the symbols in the alphabet
the automaton can read. While those automata are typically defined over semirings,
our implementation will focus for simplicity on automata with 
transitions defined over the real number field. 
The current configuration of a finite weighted automaton $W$,
defined on $n$ states, will be represented with a column vector of length
$n$, with values over the semiring or field on which transition weights of the
automaton $W$ are also defined. 

The goal of this work is to provide an high-performance implementation 
in the Go programming language of two different techniques 
to compute \textit{weighted language equivalence}.
Such equivalence relation is a bisimulation:
a relation $R$ is a bisimulation whenever two states $v_1, v_2$ in $R$ 
can simulate each other, resulting in a pair that is still in $R$.
Two state vectors $v_1, v_2$ in a weighted automaton are said to be 
weighted language equivalent, written as $v_1 \sim_l v_2$, when 
they simulate each other by accepting the same words with the same resulting output weights. 

The first technique we implement, 
defined in \cite{DBLP:journals/corr/Bonchi0K17}, is an up-to technique
for weighted language equivalence called HKC. It is defined for
weighted systems over arbitrary semirings and can be implemented with 
set theoretic constructs. The second technique is defined in \cite{BONCHI201277}:
a coalgebraic perspective is adopted to define a technique for language equivalence 
which exploits the linear representation of an automaton. 
This latter technique "\textit{minimizes}": by linearizing the state space of a
weighted automaton, 
it computes a basis for an entire linear relation (see definition \ref{def:linrel})
which coincides
with weighted language equivalence.
This technique for finite weighted automata over fields was first 
introduced by Michele Boreale in \cite{boreale2009weighted}.

Another example of the comparison between algorithms to compute 
language equivalence, precisely between HKC and an alternative
algorithm called the antichain algorithm (\cite{de2006antichains}),
was published in 2017 \cite{fu2017equivalence}.
