\section{Conclusion and Future Work}


During the testing phase, we have found that on randomly generated automata 
over the $\R$ field, the probability of BPR returning an empty basis of $\llwb$
increases together with the number of states in the automaton, and the number of
 symbols in the alphabet.
In a given linear weighted automaton $L = (V, o,t )$
over the $\R$ field and an alphabet $A$, the chance of BPR returning an empty basis also
increases together with the maximum weight in modulo of the transition matrices,
$\forall a \in A, \; \max |(t_a)_{ij}|$, for $i,j=1\hdots,n$, with 
$n = \dim(V)$. Authors of \cite{BONCHI201277} have confirmed
that it is normal for this to happen and that this fact is not due
to an implementation error.

Since the tests in this work rely heavily on a non-empty basis of $\llwb$ to generate
language equivalent vectors, the automata with an empty $\mker{\llwb}$ had to be ignored during the 
tests. This introduced substantial overhead as the chance of BPR returning an empty basis 
grew closer to 1, together with the various parameter of the automata: 
most of the computing time was spent on generating and running BPR on automata which did
not contain any language equivalent vectors, making tests on large automata not possible.

To provide better runtime results for the language equivalence problem, 
a topic worth further attention is the development of a semi-randomized method to 
generate structured large sized automata that have a low chance of BPR
returning an empty basis of $\llwb$.

Another topic worth further investigation is the cause of the 
sudden drop at 1e-15 in the F1 score plots.


An idea for the future of this implementation, is the addition
of a \texttt{Semiring} interface type that would permit much more insightful analysis
on weighted automata, not only on the field of real numbers.  

To further verify of the algorithms implemented in this work, 
and any additional algorithm that may be implemented, 
one could compare the results of this package with the results 
of the PAWS tool for the analysis of weighted systems \cite{konig2017paws}, developed 
at the University of Duisburg-Essen.
