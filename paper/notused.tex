\subsection{Calculating the intersection of linear subspaces}

To compute algorithm (\ref{fwd}), we need an efficient way to calculate a basis 
of the intersection of an arbitrary number of linear subspaces of $\R^n$, given their 
spanning sets. 

\begin{defn}
  \textbf{Algorithm for a basis of the intersection of subspaces} \\
  Defined in file \texttt{lin/intersect.go}.
  Let's consider two linear subspaces of 
  $\R^n$ defined with their spanning sets of column vectors: 
  $U = \spanset{u_1, u_2, \hdots, u_p}$ 
  and $W = \spanset{w_1, w_2, \hdots, w_k}$.
  We create the block matrix $A \in \R^{n \times (p+k)}$ defined as follows:
  
  \begin{equation*}
    \begin{aligned}
      a_{ij} = \begin{cases}
        (u_j)_i & \text{for  } j = 1, \hdots, p \quad i = 1, \hdots, n \\
        (w_{j-p})_i & \text{for  } j = p+1, \hdots, p+k \quad i = 1, \hdots, n \\
      \end{cases}
    \end{aligned}
  \end{equation*}
  
  The matrix will have the form
  \begin{equation*}
    A = \begin{bmatrix}
      \vvvec{}{u_1}{\,}, \hdots \vvvec{}{u_p}{\,}, 
      \vvvec{}{w_1}{\,}, \hdots \vvvec{}{w_k}{\,} 
    \end{bmatrix}
  \end{equation*}

  It follows that $\mker{A}$ is a basis for $U \cap W$. This can be done with an 
  arbitrary finite number of subspaces of $\R^n$, as long as their spanning set is 
  \textit{finite}.
\end{defn}

\begin{defn}
  \textbf{Algorithm for computing the null space of a vector subspace} \\
  The algorithm implementation can be found in file \texttt{lin/nullspace.go}; it is
  adapted from \cite{scipy/ranknullspace}. 
  Let's consider the singular value decomposition of a matrix $A \in \R^{m \times n}$:

  \begin{equation*}
    \begin{aligned}
      A = U \Sigma V^T & \quad & \Sigma = \diag{\sigma_1, \sigma_2, \hdots, \sigma_{\min(m, n)}  } 
       & \quad &  U \in \R^{m \times m} & \quad & V \in \R^{n \times n}
    \end{aligned}
  \end{equation*}

  Where $V$ and $U$ are orthogonal and the singular values are ordered: $\sigma_1 \geq \sigma_2 \geq \hdots \geq \sigma_{\min(m,n)} \geq 0$.
  It follows that $\mrank{A}$ is equal to the number of nonzero singular values and
  a basis of the (right) null space of $A$ is the spanning set of the columns of V
  corresponding to singular values equal to zero. 
\end{defn}

\begin{exmp}
  First, we show a shorter Python 
  implementation of the algorithm to compute the nullspace, using the 
  \textit{SciPy} library \cite{scipy/ranknullspace}:

  \lstinputlisting[language=python]{nullspace.py}

    The Go implementation is quite longer:

    \lstinputlisting[language=go]{../lin/nullspace.go}

\end{exmp}
