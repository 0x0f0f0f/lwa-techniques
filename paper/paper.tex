\documentclass{article}


\usepackage{arxiv}

\usepackage[utf8]{inputenc} % allow utf-8 input
\usepackage[T1]{fontenc}    % use 8-bit T1 fonts
\usepackage{hyperref}       % hyperlinks
\usepackage{url}            % simple URL typesetting
\usepackage{booktabs}       % professional-quality tables
\usepackage{amsfonts}       % blackboard math symbols
\usepackage{nicefrac}       % compact symbols for 1/2, etc.
\usepackage{microtype}      % microtypography
\usepackage{lipsum}
\usepackage{amsthm}
\usepackage{amsmath}
\usepackage{eucal}
\usepackage{mathtools}
\usepackage[table]{xcolor}
\usepackage{stmaryrd}
\usepackage{amsfonts}
\usepackage{lstlistings-golang}
\usepackage{caption}

% Code with syntax highlighting
\usepackage{fancyvrb}
\usepackage{listingsutf8}
\usepackage{textcomp}
\usepackage{placeins}


\theoremstyle{plain}% default
\newtheorem{thm}{Theorem}[section]
\newtheorem{lem}[thm]{Lemma}
\newtheorem{prop}[thm]{Proposition}
\newtheorem*{cor}{Corollary}
\theoremstyle{definition}
\newtheorem{defn}{Definition}[section]
\newtheorem{conj}{Conjecture}[section]
\newtheorem{exmp}{Example}[section]
\newtheorem{exrc}[exmp]{Exercise}
\theoremstyle{remark}
\newtheorem*{comm}{Comment}
\newtheorem*{note}{Note}
\newtheorem{case}{Case}

\def\C{\mathbb{C}}
\def\N{\mathbb{N}}
\def\Q{\mathbb{Q}}
\def\R{\mathbb{R}}
\def\Z{\mathbb{Z}}
\def\K{\mathbb{K}}

\DeclarePairedDelimiter\sqmap{\llbracket}{\rrbracket}
\DeclarePairedDelimiter\spanset{\langle}{\rangle}


\newcommand{\diag}[1]{\text{diag}\left(#1\right)}
\newcommand{\mspan}[1]{\text{span}\left(#1\right)}
\newcommand{\dom}{\text{dom}}
\newcommand{\mker}[1]{\text{ker}\left(#1\right)}
\newcommand{\mrank}[1]{\text{rank}\left(#1\right)}
\newcommand{\llwb}{\approx_{\mathcal{L}}}
\newcommand{\langlwa}[2]{\sqmap{#1}^\mathcal{L}_{#2}}

% matrix commands
\newcommand{\vvec}[2]{
	\begin{pmatrix}
		#1 \\ #2
	\end{pmatrix}
}
\newcommand{\vvvec}[3]{
	\begin{pmatrix}
		#1 \\ #2 \\ #3
	\end{pmatrix}
}
\newcommand{\vvvvec}[4]{
	\begin{pmatrix}
		#1 \\ #2 \\ #3 \\ #4
	\end{pmatrix}
}

% Code Listings
\definecolor{vgreen}{RGB}{104,180,104}
\definecolor{vblue}{RGB}{49,49,255}
\definecolor{vorange}{RGB}{255,143,102}
\lstset{
  numbers=left,
  frame=tb,
  commentstyle=\color{mygreen},
	basicstyle=\ttfamily,
	columns=fullflexible,
  keepspaces=false,
  showstringspaces=false,
  extendedchars=true,
  mathescape=true,
  breaklines, breakatwhitespace=true
}

\title{DRAFT: Go Implementation of Up-To Techniques for Equivalence of Weighted Languages}

%\thanks{Use footnote for providing further
%information about author (webpage, alternative
%address)---\emph{not} for acknowledging funding agencies.}

\author{
  Alessandro Cheli \\
  Undergraduate Student \\
  Department of Computer Science \\
  Università di Pisa\\
  Pisa, PI 56127 \\
  \texttt{a.cheli6@studenti.unipi.it} \\
  %% examples of more authors
  %% \And
  %% Elias D.~Striatum \\
  %% Department of Electrical Engineering\\
  %% Mount-Sheikh University\\
  %% Santa Narimana, Levand \\
  %% \texttt{stariate@ee.mount-sheikh.edu} \\
}

\definecolor{mygreen}{rgb}{0,0.6,0}



\begin{document}
\maketitle

\begin{abstract}
Weighted automata generalize non-deterministic automata by adding
a quantity expressing the weight (or probability) of the execution of each transition.
In this work we propose an implementation of two algorithms for computing language 
equivalence in finite state weighted automata (WAs). The first, a
linear partition refinement algorithm, computes the largest linear weighted bisimulation
for any given LWA (Linear Weighted Automaton) through an iterative method, 
the second algorithm checks the language equivalence 
of two vectors (states) for a given weighted automata by using an additional
data structure representing a congruence relation.
We then compare results of the two algorithms to verify their correctness
and performance on randomly generated samples. 
We finally provide comparison of runtime statistics and suggest 
which of the two algorithm is the best choice for different usage cases.
\end{abstract}


% keywords can be removed
\keywords{First keyword \and Second keyword \and More}


\section{Introduction}
\label{sec:intro}
TODO da fare

Checking whether two nondeterministic finite automata (NFA) accept the same
language is important in many application domains such as compiler construction and model checking. 
Unfortunately, solving this problem is costly: it is
PSPACE-complete (TODO cita) 
% TODO cita

In \cite{DBLP:journals/corr/Bonchi0K17}, up-to techniques are defined for
weighted systems over arbitrary semirings, while in \cite{BONCHI201277}, up-to techniques
are defined for Linear Weighted Automata (LWAs), under a more abstract coalgebraic perspective.

A well detailed example of the comparison of algorithms to compute 
language equivalence, precisely between \texttt{HKC} and an alternative
algorithm called the \textit{antichain algorithm} (\cite{de2006antichains}),
was published in 2017 \cite{fu2017equivalence}.


\section{Preliminaries and Notation}
\label{sec:notation}

\begin{note}
  Given two vector spaces $V_1, V_2$ we write $V_1 + V_2$ to 
  denote $\mspan{V_1 \cup V_2}$
\end{note}


\begin{defn}
  A \textit{weighted automaton} over a field $\K$ and an alphabet $A$ is a triple 
  $(X,o,t)$ such that $X$ is a finite set of states, 
  $t = \left\lbrace t_a : X \to \K^X\right\rbrace_{a \in A}$
  is a set of transition functions indexed over the symbols of the alphabet $A$ and 
  $o : X  \to \K$ is the output function. 
  The transition functions will be represented as $X \times X$ matrices.
  $A^*$ is the set of all words over $A$, more precisely the free monoid
  with string concatenation as the monoid operation and the empty word $\epsilon$ 
  as the identity element. We denote with $aw$ the
  concatenation of a symbol $a$ to the word $w \in A^*$.
  A weighted language is a function $\psi: A^* \to \K$.
  A function mapping each state vector into its 
  accepted language, $\sqmap{\cdot}: \K^X \to \K^{A^*}$ is defined as follows for 
  every weighted automaton:

  \begin{equation*}
    \begin{aligned}
      \forall v \in \K^X, a \in A, w \in A^* \quad \quad
      \sqmap{v}(\epsilon) = o(v) \quad \quad
      \sqmap{v}(aw) = \sqmap{t_a(v)}(w)  
    \end{aligned}
  \end{equation*}
\end{defn}

Two vectors $v_1, v_2 \in \K^{X\times 1}$ are called weighted language equivalent, 
denoted with  $v_1 \sim_l v_2 $ if and only if 
$ \sqmap {v_1} = \sqmap{v_2}$. One can extend the notion of language 
equivalence to states rather than for vectors by assigning 
to each state $x \in X$ the corresponding  unit vector 
$e_x \in \K^X$. When given an initial state $i$ for a weighted automaton, 
the language  of the automaton can be defined as $\sqmap{i}$.


\begin{defn}
  A binary relation $R \subseteq X \times Y$ between two sets $X, Y$ is a 
  subset of the 
  cartesian product of the sets. A relation is called \textit{homogeneous} 
  or an \textit
  {endorelation} if it is a binary relation over $X$ and itself: 
  $R \subseteq X \times
   X$. 
  In such case, it is simply called a binary relation over $X$.
  An \textit{equivalence relation} is a binary relation that is reflexive, 
  symmetric and
  transitive. 
\end{defn}

\begin{defn}
  The congruence closure $c(R)$ of a relation R is the smallest congruence relation 
  $R'$ such that $R \subseteq R'$ 
\end{defn}

An equivalence relation which is compatible with all the operations of
the algebraic structure on which it is defined on, is called a 
\textit{congruence relation}. Compatibility with the algebraic structure operations
means that algebraic operations applied on equivalent elements will still
yield equivalent elements. 



%===================================================================================

We omit the coalgebraic definition for \textit{linear weighted automata} seen in 
\cite{BONCHI201277} and give a more intuitive definition, which fits our  
implementation when $\K = \R$.
In this implementation, we focus only on weighted automata defined over 
the field of real numbers $\R$. 

\begin{defn}
  A \textit{linear weighted automaton} (in short, LWA) over the field $\K$ and 
  an alphabet $A$
  is a triple  $L = (V, o, \left\lbrace t_a \right\rbrace_{a \in A})$ 
  where $V$ is a vector space representing the state space, 
  $o: V \to \K$ is a linear map associating to each state its output weight,
  and $t = \left\lbrace t_a = V \times V \right\rbrace_{a \in A}$ is
  the set of transition functions, represented with liner maps 
  that for each input $a \in A$ associate the next state, in this case a vector
  in $V$.
  As in \cite{boreale2009weighted}, we have that $\dim{(L)} = \dim{(V)}$.
\end{defn}

Given a weighted automaton, one can build a corresponding linear weighted automaton
by considering the free vector space generated by the set of states $X$ in the WA,
and by linearizing $o$ and $t$. If $X$ is finite, as in our implementations
of the algorithms, we can use the same matrices for 
$t$ and $o$ in both the WA and the corresponding LWA.
We are only considering a finite number of states and therefore finite dimensional
vector spaces. Let $n$ be the number of states in an WA.
We have that in the corresponding LWA, the transition functions $t_a$ are still
 represented as
$\K^{n \times n}$ matrices. $o \in \K^{1 \times n}$ is represented as a row vector.
$t_a(v)$ denotes the vector obtained by multiplying the matrix $t_a$ by the column 
vector $v  \in \K^{n \times 1}$. $o(v)$ denotes the scalar $s \in \K$ obtained by 
dot product of the row vector $o$ with $v \in \K^{n \times 1}$.

\begin{defn}
  The language recognized by a vector $v \in V$ of an LWA $(V,o,t)$ is defined
  for all words $w \in A*$ as $\langlwa{v}{V}(w) = o(v_n)$ where $v_n$ is the 
  vector reached from $v$ through the composition of the transition functions
  corresponding to the words in $w$.
  
  \begin{equation*}
    \begin{aligned}
      \langlwa{v}{V}(w) = \begin{cases}
        o(v) & \text{if } w = \epsilon \\ 
        \langlwa{t_a(v)}{V}(w') & \text{if } w = aw' 
      \end{cases}
    \end{aligned}
  \end{equation*}
  
\end{defn}


We define $\llwb$ as the behavioral equivalence for a given LWA $(V, o, t)$ as 

\begin{equation}
  \forall v_1, v_2 \in V, \; v_1 \llwb v_2 \iff \langlwa{v_1}{V} = \langlwa{v_2}{V}
\end{equation}



Proof is available in section 3.3 of \cite{BONCHI201277}

Language equivalence can be now expressed in terms of linear weighted 
bisimulations (LWBs for short).
Differently from weighted bisimulations, LWBs can be seen both as relations 
and as subspaces.
The subspace representation of LWBs is used in the backwards partition 
refinement algorithm 
implemented in \cite{BONCHI201277} and in this work.

\begin{defn}
  \textit{Linear Relations:}\\
  Let $U$ be a subspace of $V$. The binary relation $R_U$ over $V$ is defined by
  \begin{equation*}
    \begin{aligned}
      v_1 \; R_U \; v_2 \quad \iff \quad v_1 - v_2 \in U
    \end{aligned}
  \end{equation*}
  The relation $R$ is linear if there exists a subspace $U$ such that $R \equiv R_U$.
  A linear relation is a total equivalence relation on $V$.
\end{defn}

\begin{defn}
  \textit{Kernel of a Relation and Linear Extension} \\
  Let $R$ be a binary relation over V. 
  The \textit{kernel} of $R$, is the set 
  $\mker{R} = \left\lbrace v_1 - v_2 \mid v_1 \; R \; v_2 \right\rbrace$.
  The \textit{linear extension} of $R$, written as $R^\ell$, is defined by 
  \begin{equation*}
    \begin{aligned}
      v_1 \; R^\ell \; v_2  \quad \iff \quad (v_1 - v_2) \in \mspan{\mker{R}}
    \end{aligned}
  \end{equation*}
\end{defn}

\begin{lem}
  Let $U$ be a subspace of $V$, then $\mker{R_U} = U$
\end{lem}

\begin{defn}
  \textit{Linear Weighted Bisimulation:} \\
  Let $(V, o, t)$ be a linear weighted automaton. A linear relation 
  $R \subseteq V \times V$ is a \textit{linear weighted bisimulation} if 
  $\forall (v_1, v_2) \in R$ it holds that: 
  \begin{center}
    \begin{enumerate}
      \item $o(v_1) = o(v_2)$
      \item $\forall a \in A, \; t_a(v_1) \; R \; t_a(v_2)$
    \end{enumerate}
  \end{center}
\end{defn}

\begin{lem}
  Let $(V, o, t)$ be a linear weighted automaton. A linear relation 
  $R$ over $V$ is a linear weighted bisimulation if and only if
\end{lem}
  \begin{center}
    \begin{enumerate}
      \item $R \subseteq \mker{o}$
      \item $R$ is $t_a$-invariant $\forall a \in A$
    \end{enumerate}
  \end{center}

Theorem 3 in section 3.3 of \cite{BONCHI201277}, states that 
$\mker{\langlwa{-}{V}}$ is the largest linear weighted bisimulation on $V$.
As a corollary, we obtain that $\llwb$ is the largest linear weighted bisimulation.

We now introduce a lemma that will be fundamental in the next sections of this work.

\begin{lem}
  \label{lem:coincide}
  $\llwb$ coincides with $\sim_l$: 

  Let $(X, o, t)$ be a WA and $(\K^X, o^\sharp, t^\sharp)$ the corresponding linear 
  weighted automaton. Then $\forall x \in X, \;\; \sqmap{x} = \langlwa{x}{\K^X}$
\end{lem}

%=========================================================================


\section{Algorithms}
The first algorithm we implement to compute language equivalence, called \texttt{HKC},
is adapted from \cite{DBLP:journals/corr/Bonchi0K17}. 
The algorithm returns \texttt{true} $\iff \sqmap{v_1} = \sqmap{v_2}$.
It was first introduced by 
Bonchi and Pous in \cite{bonchi2013checking}.
The algorithm, extending the Hopcroft and Karp procedure 
\cite{hopcroft1971linear} with \textit{congruence closure}, is 
proven to be sound and complete.
It is defined for 
WAs over semirings, but in this implementation we are only 
considering fields, in particular 
the field of real numbers ($\K = \R$).
The problem of checking language equivalence 
has been proven undecidable for an arbitrary semiring, so termination 
may not always be guaranteed. However, it has been shown to be decidable
for a broad range of semirings, for example, all the complete and
distributive lattices.
\texttt{HKC} computes $v_1 \sim_l v_2$ for a given weighted automaton
$W = (X, t, o)$ and two vectors $v_1, v_2 \in \K^X$. 
by computing a congruence closure,
and it does so without linearizing the state space. 

We compare \texttt{HKC} with an algorithm called 
\textit{Backwards Partition Refinement}, that we will call BPR for short. 
Adapted from \cite{BONCHI201277}, 
\texttt{BPR}
is a fixed-point iterative method that, given an LWA
$L = (V, t, o)$, computes a basis of the subspace of $V$
representing the complementary relation of $\llwb$ (we later show it to be the 
orthogonal complement in case $V$ is an inner product space). 
Another version of the algorithm is defined in the same work,
called \textit{Forward Partition Refinement}, which directly computes
a basis for $\llwb$ but is shown to be way less efficient than the backwards version.

Our implementation is directly modeled on the algorithm shown in 
\cite{boreale2009weighted}, since we are fixing weights on $\R$ and computing the orthogonal complements instead of 
dual spaces and annihilators. 

\begin{note}
  Recall from section \ref{sec:notation} that $\llwb$ is a linear relation, 
  therefore $v_1 \llwb v_2 \iff (v_1 - v_2) \in \mker{\llwb}$
\end{note}


From Lemma \ref{lem:coincide}, it follows that given an LWA $L = (V,o,t)$ and a corresponding
basis of $\llwb$ computed by BPR, one can check language equivalence 
of two vectors in the state space, $v_1 \sim_l v_2 $,  by checking if $(v_1 - v_2) \in \mker{\llwb}$.
Therefore, we can say that BPR "\textit{minimizes}", or it computes the whole
binary linear relation $\llwb$, coinciding with $\sim_l$.



The \texttt{BPR} algorithm starts from the basis of a relation $R_0$, that is the complement 
of the relation identifying vectors with equal weights.
It then incrementally computes the space of all states that are reachable from 
$R_0$ in a \textit{backwards} direction. Intuitively, "going backwards" means 
working with the transpose transitions functions $t_a^T$.

\texttt{BPR} has a cost of $O(n^4)$ operations 
to initially compute the largest linear weighted bisimulation,
which can be eventually reduced to $O(n^3)$ \cite{BONCHI201277}.
In our implementation, by initially computing a basis of the orthogonal complement of $\llwb$,
the cost of checking if two vectors are language equivalent is then reduced to the
cost of matrix multiplication ($O(n^2)$). It follows that 
\texttt{BPR} is a great choice when we
have to decide if a large number of vectors in a WA are language equivalent.


In the next sections we will compare execution results of our implementation of the algorithms
\texttt{BPR} and \texttt{HKC} to verify correctness,
and to provide insight on runtime results.
Lemma \ref{lem:coincide}, introduced above, is key to our work. By stating that 
$\llwb$ coincides with $\sim_l$, we can confidently say that the two algorithms 
compute an answer for same the decision problem:

\begin{center}
    Are two vectors $v_1$ and $v_2$ language-equivalent for a given weighted automata? 
\end{center}

%TODO
%TODO costo computazionale HKC.


%=========================================================================

\subsection{HKC Algorithm}
We give the pseudocode definition of the \texttt{HKC} procedure from \cite{DBLP:journals/corr/Bonchi0K17}:

\captionof{figure}{The \texttt{HKC}($v_1, v_2$) procedure}
\lstinputlisting[escapeinside={(*}{*)}]{hkc.txt}
\label{fig:hkc}

\subsection{Backwards Partition Refinement Algorithm for the Largest Weighted Bisimulation}
\label{sec:algo2}

We now recall the backwards algorithm for computing $\llwb$ defined in \cite{BONCHI201277}.
The algorithm is defined by the iterative method:
\begin{eqnarray}
  R_0 = \mker{o}^0, & \quad & R_{i+1} = R_i + \sum_{a \in A} t_a^T(R_i) \label{back} 
\end{eqnarray}
Where $\mker{o}^0$ is an annihilator.

\begin{note}
  Given two vector spaces $V_1, V_2$ we write $V_1 + V_2$ to 
  denote $\mspan{V_1 \cup V_2}$
\end{note}


The algorithm stops when $R_{j+1} = R_j$. An index $j \leq \dim(V)$ is 
guaranteed to exist, such that the algorithms stops at step $j$.
It follows that $\llwb = R_j^0$.
Proof is available in section 4.2 of \cite{BONCHI201277}





\section{Implementation}
\label{sec:impl}

The algorithms and data structures for this paper are implemented in the Go programming 
language. This implementation makes use of the Gonum library for numerical 
computations. We only import the Gonum libraries for matrices and linear algebra 
and visual plotting of samples and functions.
Real numbers are implemented with double precision floating point numbers,  
known as the \texttt{float64} type in the Go programming language.


\begin{defn}
  \textbf{Applications of SVD} \\

  Let's consider the singular value decomposition of a matrix $A \in \R^{m \times n}$:

  \begin{equation*}
    \begin{aligned}
      A = U \Sigma V^T & \quad & \Sigma = \diag{\sigma_1, \sigma_2, \hdots, \sigma_r  } 
       & \quad &  U \in \R^{m \times m} & \quad & V \in \R^{n \times n}
    \end{aligned}
  \end{equation*}

  Where $V$ and $U$ are orthogonal and the singular values are ordered: $\sigma_1 \geq \sigma_2 \geq \hdots \geq \sigma_r \geq 0$.
  It follows that $\mrank{A}$ is equal to the number of nonzero singular values, and
  as explained in \cite{svd}:
  
  \begin{enumerate}
    \item  $\mrank{A} = \mrank{\Sigma} = r$
    \item The column space of $A$ is spanned by the first $r$ columns of $U$.
    \item The null space of $A$ is spanned by the last $n − r$ columns of $V$.
    \item The row space of $A$ is spanned by the first $r$ columns of $V$.
    \item The null space of $A^T$ is spanned by the last $m − r$ columns of $U$.
  \end{enumerate}
  
  Of our interest, are only the computation of the null space and columns space.
  The implementation, applying SVD, 
  can be found in files \texttt{lin/colspace.go} and \texttt{lin/nullspace.go}.
\end{defn}






\subsection{Implementing the backwards partition refinement algorithm}

To compute $\llwb$ at the last step of the algorithm,
we need to compute $R_j^0$.
If $V$ is a vector space and $W$ is a
subspace of $W$, the annihilator of $W$, respectively $W^0$ is 
a subspace of the space $V^*$ of linear functionals on $V$.
$W^0$ are the functionals that annihilate on $W$. Since 
we are working on subspaces of $\R^n$, we can directly compute 
the orthogonal complement in our implementation instead of the
annihilator.


\begin{prop}
  If $V$ is a finite dimensional vector space defined with an inner product
  $\langle \cdot , \cdot \rangle$ and $W$ is a subspace of $V$
  then the image of the annhilitaor $W^0$ through the linear 
  isomorphism $\varphi: V^* \to V$ induced by the inner product, 
  is the orthogonal of $W$ with respect to the said inner product.
\end{prop}

\begin{proof}
  Let $V$ be an inner product space over the field $\K$ with an inner product defined as
  $\langle \cdot , \cdot \rangle : V \times V \to \K$. 
  Every linear functional can be 
  represented with a vector. Let $\xi : V \to \K$ be a functional, 
  $\xi \in  W^0$. Because $\xi(w)=0 \quad \forall  w \in W$, 
  if $v$ represents $\xi$ we have that $(v, w)=\xi(w)=0$ for all $w \in W$. 
  We obtain that $\varphi(W^0) \subseteq W^{\perp}$.
  If $v \in W^\perp$  
  then the functional $x \mapsto (v, x)$ cancels over $W$ 
  (by the definition of orthogonality).
\end{proof}


To compute the orthogonal complement of a vector subspace $W$, we
compute $W^\perp = \mker{A^T}$, where $A$ is the matrix with 
column vectors in the spanning set of $W$ as its columns. Precisely, $W$ is 
represented as the 
column space of $A$. Proof is available in \cite{ila}.


%=============================================================


%\begin{figure}
%  \centering
%  \fbox{\rule[-.5cm]{4cm}{4cm} \rule[-.5cm]{4cm}{0cm}}
%  \caption{Sample figure caption.}
%  \label{fig:fig1}
%\end{figure}


%\begin{table}
% \caption{Sample table title}
%  \centering
%  \begin{tabular}{lll}
%    \toprule
%    \multicolumn{2}{c}{Part}                   \\
%    \cmidrule(r){1-2}
%    Name     & Description     & Size ($\mu$m) \\
%    \midrule
%    Dendrite & Input terminal  & $\sim$100     \\
%    Axon     & Output terminal & $\sim$10      \\
%    Soma     & Cell body       & up to $10^6$  \\
%    \bottomrule
%  \end{tabular}
%  \label{tab:table}
%\end{table}





\bibliographystyle{unsrt}
\bibliography{references}  %%% Remove comment to use the external .bib file (using bibtex).
%%% and comment out the ``thebibliography'' section.

% 
% %%% Comment out this section when you \bibliography{references} is enabled.
% \begin{thebibliography}{1}
% 
% \bibitem{kour2014real}
% George Kour and Raid Saabne.
% \newblock Real-time segmentation of on-line handwritten arabic script.
% \newblock In {\em Frontiers in Handwriting Recognition (ICFHR), 2014 14th
%   International Conference on}, pages 417--422. IEEE, 2014.
% 
% \bibitem{kour2014fast}
% George Kour and Raid Saabne.
% \newblock Fast classification of handwritten on-line arabic characters.
% \newblock In {\em Soft Computing and Pattern Recognition (SoCPaR), 2014 6th
%   International Conference of}, pages 312--318. IEEE, 2014.
% 
% \bibitem{hadash2018estimate}
% Guy Hadash, Einat Kermany, Boaz Carmeli, Ofer Lavi, George Kour, and Alon
%   Jacovi.
% \newblock Estimate and replace: A novel approach to integrating deep neural
%   networks with existing applications.
% \newblock {\em arXiv preprint arXiv:1804.09028}, 2018.
% 
% \end{thebibliography}
% 
% 
\appendix

%\section{Complete Code}

\subsection{Linear Algebra Utilities}

\captionof{figure}{\texttt{lin/util.go}}
\lstinputlisting[language=go]{../lin/util.go}

\captionof{figure}{\texttt{lin/subspace.go}}
\lstinputlisting[language=go]{../lin/subspace.go}

\captionof{figure}{\texttt{lin/nullspace.go}}
\lstinputlisting[language=go]{../lin/nullspace.go}

\captionof{figure}{\texttt{lin/colspace.go}}
\lstinputlisting[language=go]{../lin/colspace.go}


\subsection{Automata Data Structures and Methods}

\captionof{figure}{automata/automata.go}
\lstinputlisting[language=go]{../automata/automata.go}

\captionof{figure}{automata/methods.go}
\lstinputlisting[language=go]{../automata/methods.go}

\captionof{figure}{automata/io.go}
\lstinputlisting[language=go]{../automata/io.go}

\captionof{figure}{automata/random.go}
\lstinputlisting[language=go]{../automata/random.go}

\subsection{Pair, Relation and To-Do list Structures and Methods}


\captionof{figure}{automata/pair.go}
\lstinputlisting[language=go]{../automata/pair.go}

\captionof{figure}{automata/pairstack.go}
\lstinputlisting[language=go]{../automata/pairstack.go}

\captionof{figure}{automata/relation.go}
\lstinputlisting[language=go]{../automata/relation.go}


\subsection{Algorithms}

\captionof{figure}{automata/hkc.go}
\lstinputlisting[language=go]{../automata/hkc.go}

\captionof{figure}{automata/backwards.go}
\lstinputlisting[language=go]{../automata/backwards.go}


\subsection{Random Batch Tests Package}

\captionof{figure}{\texttt{randtest/data.go}}
\lstinputlisting[language=go]{../randtest/data.go}

\captionof{figure}{\texttt{randtest/randtest.go}}
\lstinputlisting[language=go]{../randtest/randtest.go}


\captionof{figure}{\texttt{randtest/f1-tol.go}}
\lstinputlisting[language=go]{../randtest/f1-tol.go}


\captionof{figure}{\texttt{main.go}}
\lstinputlisting[language=go]{../main.go}


%\section{Test Files}

%\captionof{figure}{\texttt{lin/subspace_test.go}}
%\lstinputlisting[language=go]{../lin/subspace_test.go}


%\captionof{figure}{automata/backwards_test.go}
%\lstinputlisting[language=go]{../automata/backwards_test.go}

\end{document}
 