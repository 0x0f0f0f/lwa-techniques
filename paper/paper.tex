\documentclass{article}


\usepackage{arxiv}

\usepackage[utf8]{inputenc} % allow utf-8 input
\usepackage[T1]{fontenc}    % use 8-bit T1 fonts
\usepackage{hyperref}       % hyperlinks
\usepackage{url}            % simple URL typesetting
\usepackage{booktabs}       % professional-quality tables
\usepackage{amsfonts}       % blackboard math symbols
\usepackage{nicefrac}       % compact symbols for 1/2, etc.
\usepackage{microtype}      % microtypography
\usepackage{lipsum}
\usepackage{amsthm}
\usepackage{mathtools}
\usepackage{stmaryrd}
\usepackage{amsfonts}


\theoremstyle{plain}% default
\newtheorem{thm}{Theorem}[section]
\newtheorem{lem}[thm]{Lemma}
\newtheorem{prop}[thm]{Proposition}
\newtheorem*{cor}{Corollary}
\theoremstyle{definition}
\newtheorem{defn}{Definition}[section]
\newtheorem{conj}{Conjecture}[section]
\newtheorem{exmp}{Example}[section]
\newtheorem{exrc}[exmp]{Exercise}
\theoremstyle{remark}
\newtheorem*{comm}{Comment}
\newtheorem*{note}{Note}
\newtheorem{caso}{Case}

\def\C{\mathbb{C}}
\def\N{\mathbb{N}}
\def\Q{\mathbb{Q}}
\def\R{\mathbb{R}}
\def\Z{\mathbb{Z}}
\def\K{\mathbb{K}}

\DeclarePairedDelimiter\sqmap{\llbracket}{\rrbracket}


% matrix commands
\newcommand{\vvec}[2]{
	\begin{bmatrix}
		#1 \\ #2
	\end{bmatrix}
}
\newcommand{\vvvec}[3]{
	\begin{bmatrix}
		#1 \\ #2 \\ #3
	\end{bmatrix}
}


\title{Go Implementation of Up-To Techniques for Equivalence of Weighted Languages}

%\thanks{Use footnote for providing further
%information about author (webpage, alternative
%address)---\emph{not} for acknowledging funding agencies.}

\author{
  Alessandro Cheli \\
  Undergraduate Student \\
  Department of Computer Science \\
  Università di Pisa\\
  Pisa, PI 56127 \\
  \texttt{a.cheli6@studenti.unipi.it} \\
  %% examples of more authors
  %% \And
  %% Elias D.~Striatum \\
  %% Department of Electrical Engineering\\
  %% Mount-Sheikh University\\
  %% Santa Narimana, Levand \\
  %% \texttt{stariate@ee.mount-sheikh.edu} \\
  %% \AND
  %% Coauthor \\
  %% Affiliation \\
  %% Address \\
  %% \texttt{email} \\
  %% \And
  %% Coauthor \\
  %% Affiliation \\
  %% Address \\
  %% \texttt{email} \\
  %% \And
  %% Coauthor \\
  %% Affiliation \\
  %% Address \\
  %% \texttt{email} \\
}

\begin{document}
\maketitle

\begin{abstract}
Weighted automata generalize non-deterministic automata by adding
a quantity expressing the weight (or probability) of the execution of each transition.
In this work we propose an implementation of two algorithms for computing language 
equivalence in finite state linear weighted automata (LWAs). The first, a
linear partition refinement algorithm, calculates the linear weighted bisimulation
for any given LWA, the second algorithm checks the equivalence 
of two vectors (states) for a given weighted automata, and is used to verify the results
of the first algorithm. We then provide and compare runtime results. 
\end{abstract}


% keywords can be removed
\keywords{First keyword \and Second keyword \and More}


\section{Introduction}
\label{sec:intro}

In \cite{DBLP:journals/corr/Bonchi0K17}, up-to techniques are defined for
weighted systems over arbitrary semirings, while in \cite{BONCHI201277}, up-to techniques
are studied under a more abstract coalgebraic perspective. For the sake of simplicity of 
the implementation, we focus only on weighted automata over the field of real numbers 
$\R$. Real numbers are implemented with double precision floating point numbers,  
known as the \texttt{float64} type in the Go programming language.


\section{Preliminaries and Notation}
\label{sec:notation}

\begin{defn}
  A \textit{weighted automaton} over a field $\K$ and an alphabet $A$ is a triple 
  $(X,o,t)$ such that $X$ is a finite set of states, $t = (t_a : X \to K)_{a \in A}$
  is a set of transition functions indexed over the symbols of the alphabet and $o : X 
  \to \K$ is the output function. The transition functions $t_a$ are represented as
  $\K^{X \times X}$ matrices. $o \in \K^X$ is represented as a row vector.
  $t_a(v)$ denotes the vector obtained by multiplying the matrix $t_a$ by the vector $v 
  \in \K^X$. $o(v)$ denotes the scalar $s \in \K^X$ obtained by multiplying the row
  vector $o$ by the column vector $v \in \K^X$.
  $A^*$ is the set of all words over $A$. $\epsilon$ is the empty word and $aw$ is the
  concatenation of a letter $a$ to the word $w \in A^*$.
  A weighted language is a function $\psi: A^* \to \K$.
  A function mapping each state vector into its 
  accepted language, $\sqmap{\cdot}: \K^X \to K^{A^*}$ is defined as follows for every weighted automaton:

  \begin{equation*}
    \begin{aligned}
      \forall v \in \K^X, a \in A, w \in A^* \quad \quad
      \sqmap{v}(\epsilon) = o(v) \quad \quad
      \sqmap{v}(aw) = \sqmap{t_a(v)}(w)  
    \end{aligned}
  \end{equation*}

  Two vectors $v_1, v_2 \in \K^X$ are called language equivalent $v_1 \sim v_2 \iff \sqmap
  {v_1} = \sqmap{v_2}$. One can extend the notion of language equivalence to states rather
  than for vectors by assigning to each state $x \in X$ the corresponding unit vector 
  $e_x \in \K^X$. When given an initial state $i$ for a weighted automaton, the language 
  of the automaton can be defined as $\sqmap{i}$.
\end{defn}


\begin{defn}
  A binary relation $R \subseteq X \times Y$ between two sets $X, Y$ is a subset of the 
  cartesian product of the sets. A relation is called \textit{homogeneous} or an \textit
  {endorelation} if it is a binary relation over $X$ and itself: $R \subseteq X \times
   X$. 
  In such case, it is simply called a binary relation over $X$.
  An \textit{equivalence relation} is a binary relation that is reflexive, symmetric and
  transitive. 
\end{defn}
  
An equivalence relation which is compatible with all the operations of
the algebraic structure on which it is defined on, is called a 
\textit{congruence relation}. Compatibility with the algebraic structure operations
means that algebraic operations applied on equivalent elements will still
yield equivalent elements. 


\begin{defn}
  The congruence closure $c(R)$ of a relation R is the smallest congruence relation 
  $R'$ such that $R \subseteq R'$ 
\end{defn}

\begin{defn}
  \textbf{} \\
  
\end{defn}

\section{Implementation}
\label{sec:impl}

The algorithms and data structures for this paper are implemented in the Go programming 
language. This implementation makes use of the \textit{Gonum} library for numerical 
computations. We only import the Gonum libraries for matrices and linear algebra 
and visual plotting of samples and functions.

See Section \ref{sec:headings}.

\subsection{Headings: second level}


\subsubsection{Headings: third level}

\paragraph{Paragraph}

\section{Examples of citations, figures, tables, references}
\label{sec:others}

\cite{kour2014real,kour2014fast} and see \cite{hadash2018estimate}.

The documentation for \verb+natbib+ may be found at
\begin{center}
  \url{http://mirrors.ctan.org/macros/latex/contrib/natbib/natnotes.pdf}
\end{center}
Of note is the command \verb+\citet+, which produces citations
appropriate for use in inline text.  For example,
\begin{verbatim}
   \citet{hasselmo} investigated\dots
\end{verbatim}
produces
\begin{quote}
  Hasselmo, et al.\ (1995) investigated\dots
\end{quote}

\begin{center}
  \url{https://www.ctan.org/pkg/booktabs}
\end{center}


\subsection{Figures}
See Figure \ref{fig:fig1}. Here is how you add footnotes. \footnote{Sample of the first footnote.}

\begin{figure}
  \centering
  \fbox{\rule[-.5cm]{4cm}{4cm} \rule[-.5cm]{4cm}{0cm}}
  \caption{Sample figure caption.}
  \label{fig:fig1}
\end{figure}

\subsection{Tables}
See awesome Table~\ref{tab:table}.

\begin{table}
 \caption{Sample table title}
  \centering
  \begin{tabular}{lll}
    \toprule
    \multicolumn{2}{c}{Part}                   \\
    \cmidrule(r){1-2}
    Name     & Description     & Size ($\mu$m) \\
    \midrule
    Dendrite & Input terminal  & $\sim$100     \\
    Axon     & Output terminal & $\sim$10      \\
    Soma     & Cell body       & up to $10^6$  \\
    \bottomrule
  \end{tabular}
  \label{tab:table}
\end{table}

\subsection{Lists}
\begin{itemize}
\item Lorem ipsum dolor sit amet
\item consectetur adipiscing elit. 
\item Aliquam dignissim blandit est, in dictum tortor gravida eget. In ac rutrum magna.
\end{itemize}


\bibliographystyle{unsrt}  
\bibliography{references}  %%% Remove comment to use the external .bib file (using bibtex).
%%% and comment out the ``thebibliography'' section.

% 
% %%% Comment out this section when you \bibliography{references} is enabled.
% \begin{thebibliography}{1}
% 
% \bibitem{kour2014real}
% George Kour and Raid Saabne.
% \newblock Real-time segmentation of on-line handwritten arabic script.
% \newblock In {\em Frontiers in Handwriting Recognition (ICFHR), 2014 14th
%   International Conference on}, pages 417--422. IEEE, 2014.
% 
% \bibitem{kour2014fast}
% George Kour and Raid Saabne.
% \newblock Fast classification of handwritten on-line arabic characters.
% \newblock In {\em Soft Computing and Pattern Recognition (SoCPaR), 2014 6th
%   International Conference of}, pages 312--318. IEEE, 2014.
% 
% \bibitem{hadash2018estimate}
% Guy Hadash, Einat Kermany, Boaz Carmeli, Ofer Lavi, George Kour, and Alon
%   Jacovi.
% \newblock Estimate and replace: A novel approach to integrating deep neural
%   networks with existing applications.
% \newblock {\em arXiv preprint arXiv:1804.09028}, 2018.
% 
% \end{thebibliography}
% 
% 
\end{document}
 