\documentclass{article}


\usepackage{arxiv}

\usepackage[utf8]{inputenc} % allow utf-8 input
\usepackage[T1]{fontenc}    % use 8-bit T1 fonts
\usepackage{hyperref}       % hyperlinks
\usepackage{url}            % simple URL typesetting
\usepackage{booktabs}       % professional-quality tables
\usepackage{amsfonts}       % blackboard math symbols
\usepackage{nicefrac}       % compact symbols for 1/2, etc.
\usepackage{microtype}      % microtypography
\usepackage{lipsum}
\usepackage{amsthm}
\usepackage{amsfonts}


\theoremstyle{plain}% default
\newtheorem{thm}{Theorem}[section]
\newtheorem{lem}[thm]{Lemma}
\newtheorem{prop}[thm]{Proposition}
\newtheorem*{cor}{Corollary}
\theoremstyle{definition}
\newtheorem{defn}{Definition}[section]
\newtheorem{conj}{Conjecture}[section]
\newtheorem{exmp}{Example}[section]
\newtheorem{exrc}[exmp]{Exercise}
\theoremstyle{remark}
\newtheorem*{comm}{Comment}
\newtheorem*{note}{Note}
\newtheorem{caso}{Case}

\def\C{\mathbb{C}}
\def\N{\mathbb{N}}
\def\Q{\mathbb{Q}}
\def\R{\mathbb{R}}
\def\Z{\mathbb{Z}}

% matrix commands
\newcommand{\vvec}[2]{
	\begin{bmatrix}
		#1 \\ #2
	\end{bmatrix}
}
\newcommand{\vvvec}[3]{
	\begin{bmatrix}
		#1 \\ #2 \\ #3
	\end{bmatrix}
}


\title{Go Implementation of Up-To Techniques for Equivalence of Weighted Languages}

%\thanks{Use footnote for providing further
%information about author (webpage, alternative
%address)---\emph{not} for acknowledging funding agencies.}

\author{
  Alessandro Cheli \\
  Undergraduate Student \\
  Department of Computer Science \\
  Università di Pisa\\
  Pisa, PI 56127 \\
  \texttt{a.cheli6@studenti.unipi.it} \\
  %% examples of more authors
  %% \And
  %% Elias D.~Striatum \\
  %% Department of Electrical Engineering\\
  %% Mount-Sheikh University\\
  %% Santa Narimana, Levand \\
  %% \texttt{stariate@ee.mount-sheikh.edu} \\
  %% \AND
  %% Coauthor \\
  %% Affiliation \\
  %% Address \\
  %% \texttt{email} \\
  %% \And
  %% Coauthor \\
  %% Affiliation \\
  %% Address \\
  %% \texttt{email} \\
  %% \And
  %% Coauthor \\
  %% Affiliation \\
  %% Address \\
  %% \texttt{email} \\
}

\begin{document}
\maketitle

\begin{abstract}
Weighted automata generalize non-deterministic automata by adding
a quantity expressing the weight (or probability) of the execution of each transition.
In this work we propose an implementation of two algorithms for computing language 
equivalence in finite state Linear Weighted Automata (LWAs). The first, a
linear partition refinement algorithm, calculates the linear weighted bisimulation
for any given LWA, the second algorithm is to check the equivalence 
of two state vectors for a weighted automata, and is used to verify the results
from the first algorithm. We then provide and compare runtime results. 
\end{abstract}


% keywords can be removed
\keywords{First keyword \and Second keyword \and More}


\section{Introduction}
In \cite{DBLP:journals/corr/Bonchi0K17}, up-to techniques are defined for
weighted systems over arbitrary semirings, while in \cite{BONCHI201277}, up-to techniques
are studied under a more abstract coalgebraic perspective. For simplicity, this example implementation
is focused on weighted automata over the field $\R$ of real numbers, represented with double
precision floating point machine numbers.

We first fix some notation used in the paper.
\begin{defn}
  A binary relation $R$ between two sets $X, Y$ is a subset of the Cartesian Product: 
  $R \subseteq X \times Y$. A relation is called \textit{homogeneous} or an \textit{endorelation},
  if it is a binary relation over $X$ and itself: $R \subseteq X \times X$ It can be simply called a binary relation
  over $X$.
  
\end{defn}


\lipsum[2]
\lipsum[3]


\section{Headings: first level}
\label{sec:headings}

\lipsum[4] See Section \ref{sec:headings}.

\subsection{Headings: second level}
\lipsum[5]
\begin{equation}
\xi _{ij}(t)=P(x_{t}=i,x_{t+1}=j|y,v,w;\theta)= {\frac {\alpha _{i}(t)a^{w_t}_{ij}\beta _{j}(t+1)b^{v_{t+1}}_{j}(y_{t+1})}{\sum _{i=1}^{N} \sum _{j=1}^{N} \alpha _{i}(t)a^{w_t}_{ij}\beta _{j}(t+1)b^{v_{t+1}}_{j}(y_{t+1})}}
\end{equation}

\subsubsection{Headings: third level}
\lipsum[6]

\paragraph{Paragraph}
\lipsum[7]

\section{Examples of citations, figures, tables, references}
\label{sec:others}
\lipsum[8] \cite{kour2014real,kour2014fast} and see \cite{hadash2018estimate}.

The documentation for \verb+natbib+ may be found at
\begin{center}
  \url{http://mirrors.ctan.org/macros/latex/contrib/natbib/natnotes.pdf}
\end{center}
Of note is the command \verb+\citet+, which produces citations
appropriate for use in inline text.  For example,
\begin{verbatim}
   \citet{hasselmo} investigated\dots
\end{verbatim}
produces
\begin{quote}
  Hasselmo, et al.\ (1995) investigated\dots
\end{quote}

\begin{center}
  \url{https://www.ctan.org/pkg/booktabs}
\end{center}


\subsection{Figures}
\lipsum[10] 
See Figure \ref{fig:fig1}. Here is how you add footnotes. \footnote{Sample of the first footnote.}
\lipsum[11] 

\begin{figure}
  \centering
  \fbox{\rule[-.5cm]{4cm}{4cm} \rule[-.5cm]{4cm}{0cm}}
  \caption{Sample figure caption.}
  \label{fig:fig1}
\end{figure}

\subsection{Tables}
\lipsum[12]
See awesome Table~\ref{tab:table}.

\begin{table}
 \caption{Sample table title}
  \centering
  \begin{tabular}{lll}
    \toprule
    \multicolumn{2}{c}{Part}                   \\
    \cmidrule(r){1-2}
    Name     & Description     & Size ($\mu$m) \\
    \midrule
    Dendrite & Input terminal  & $\sim$100     \\
    Axon     & Output terminal & $\sim$10      \\
    Soma     & Cell body       & up to $10^6$  \\
    \bottomrule
  \end{tabular}
  \label{tab:table}
\end{table}

\subsection{Lists}
\begin{itemize}
\item Lorem ipsum dolor sit amet
\item consectetur adipiscing elit. 
\item Aliquam dignissim blandit est, in dictum tortor gravida eget. In ac rutrum magna.
\end{itemize}


\bibliographystyle{unsrt}  
\bibliography{references}  %%% Remove comment to use the external .bib file (using bibtex).
%%% and comment out the ``thebibliography'' section.

% 
% %%% Comment out this section when you \bibliography{references} is enabled.
% \begin{thebibliography}{1}
% 
% \bibitem{kour2014real}
% George Kour and Raid Saabne.
% \newblock Real-time segmentation of on-line handwritten arabic script.
% \newblock In {\em Frontiers in Handwriting Recognition (ICFHR), 2014 14th
%   International Conference on}, pages 417--422. IEEE, 2014.
% 
% \bibitem{kour2014fast}
% George Kour and Raid Saabne.
% \newblock Fast classification of handwritten on-line arabic characters.
% \newblock In {\em Soft Computing and Pattern Recognition (SoCPaR), 2014 6th
%   International Conference of}, pages 312--318. IEEE, 2014.
% 
% \bibitem{hadash2018estimate}
% Guy Hadash, Einat Kermany, Boaz Carmeli, Ofer Lavi, George Kour, and Alon
%   Jacovi.
% \newblock Estimate and replace: A novel approach to integrating deep neural
%   networks with existing applications.
% \newblock {\em arXiv preprint arXiv:1804.09028}, 2018.
% 
% \end{thebibliography}
% 
% 
\end{document}
 